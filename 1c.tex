\documentclass[11pt]{article}
\usepackage{amsmath,amssymb,amsthm}
\usepackage{geometry}
\geometry{margin=1in}
\title{Section 1C: Subspaces\\Worked Solutions}
\author{}
\date{}
\begin{document}
\maketitle

\section*{Problem Setup}
Let $F$ be a field and $V$ a vector space over $F$. For each part below, we determine whether the given set is a subspace and justify the claim.

\section*{Problem 1}
For each of the following subsets of $F^4$, determine whether it is a subspace of $F^4$.

\subsection*{(a)}
\textbf{Claim.}
$\{(x_1-x_2, x_2, x_1+x_2, x_3+x_4) \in F^4 : x_1, x_2, x_3, x_4 \in F\}$ is a subspace of $F^4$.

\textbf{Proof.}
Let $U = \{(x_1-x_2, x_2, x_1+x_2, x_3+x_4) : x_1, x_2, x_3, x_4 \in F\}$.

Note that every element of $F^4$ can be written in this form. Given any $(a, b, c, d) \in F^4$, we can solve:
\begin{align*}
x_1 - x_2 &= a \\
x_2 &= b \\
x_1 + x_2 &= c \\
x_3 + x_4 &= d
\end{align*}

From the first two equations: $x_1 = a + b$ and $x_2 = b$. Checking the third equation: $x_1 + x_2 = (a+b) + b = a + 2b$. For this to equal $c$ for all choices, we need $c = a + 2b$, which is not always true.

Actually, let me reconsider. The set $U$ consists of all vectors that can be expressed as $(x_1-x_2, x_2, x_1+x_2, x_3+x_4)$ for some choice of $x_1, x_2, x_3, x_4 \in F$. This is not all of $F^4$.

\textbf{Zero vector:} Taking $x_1 = x_2 = x_3 = x_4 = 0$ gives $(0, 0, 0, 0) \in U$.

\textbf{Closed under addition:} If $(a, b, c, d) = (x_1-x_2, x_2, x_1+x_2, x_3+x_4)$ and $(a', b', c', d') = (y_1-y_2, y_2, y_1+y_2, y_3+y_4)$, then
\begin{align*}
(a+a', b+b', c+c', d+d') &= ((x_1+y_1)-(x_2+y_2), x_2+y_2, (x_1+y_1)+(x_2+y_2), (x_3+y_3)+(x_4+y_4)),
\end{align*}
which is in $U$ with parameters $x_1+y_1, x_2+y_2, x_3+y_3, x_4+y_4$.

\textbf{Closed under scalar multiplication:} If $(a, b, c, d) = (x_1-x_2, x_2, x_1+x_2, x_3+x_4)$ and $\lambda \in F$, then
\begin{align*}
\lambda(a, b, c, d) &= (\lambda(x_1-x_2), \lambda x_2, \lambda(x_1+x_2), \lambda(x_3+x_4)) \\
&= (\lambda x_1 - \lambda x_2, \lambda x_2, \lambda x_1 + \lambda x_2, \lambda x_3 + \lambda x_4),
\end{align*}
which is in $U$ with parameters $\lambda x_1, \lambda x_2, \lambda x_3, \lambda x_4$.

Therefore $U$ is a subspace of $F^4$. In fact, $U = F^4$ since we can achieve any fourth coordinate with appropriate $x_3, x_4$, and the first three coordinates have two degrees of freedom ($x_1$ and $x_2$).

Actually, to be more precise: the map $(x_1, x_2, x_3, x_4) \mapsto (x_1-x_2, x_2, x_1+x_2, x_3+x_4)$ is a linear map from $F^4$ to $F^4$, so its image $U$ is a subspace. The map is surjective (we can solve for any target), so $U = F^4$.
\hfill $\square$

\subsection*{(b)}
\textbf{Claim.}
$\{(x_1-x_2, x_2, x_1+x_2, x_3+x_4) \in F^4 : x_1, x_2, x_3, x_4 \in F, x_3 = 0\}$ is a subspace of $F^4$.

\textbf{Proof.}
Let $U = \{(x_1-x_2, x_2, x_1+x_2, x_4) : x_1, x_2, x_4 \in F\}$ (since $x_3 = 0$ implies $x_3 + x_4 = x_4$).

This is the image of a linear map $F^3 \to F^4$ given by $(x_1, x_2, x_4) \mapsto (x_1-x_2, x_2, x_1+x_2, x_4)$, so $U$ is a subspace. We can verify directly:

\textbf{Zero vector:} $(0, 0, 0, 0) \in U$ (take $x_1 = x_2 = x_4 = 0$).

\textbf{Closed under addition and scalar multiplication:} Similar to part (a).

Therefore $U$ is a subspace.
\hfill $\square$

\subsection*{(c)}
\textbf{Claim.}
$\{(x_1-x_2, x_2, x_1+x_2, x_3+x_4) \in F^4 : x_1, x_2, x_3, x_4 \in F, x_3 = x_4\}$ is a subspace of $F^4$.

\textbf{Proof.}
Let $U = \{(x_1-x_2, x_2, x_1+x_2, 2x_3) : x_1, x_2, x_3 \in F\}$ (since $x_3 = x_4$ implies $x_3 + x_4 = 2x_3$).

This is the image of a linear map $F^3 \to F^4$, so $U$ is a subspace.
\hfill $\square$

\subsection*{(d)}
\textbf{Claim.}
$\{(x_1-x_2, x_2, x_1+x_2, x_3+x_4) \in F^4 : x_1 = x_2 = 5x_3\}$ is not a subspace of $F^4$ unless the characteristic of $F$ is 2.

\textbf{Proof.}
If $x_1 = x_2 = 5x_3$, then the vectors have the form:
\begin{align*}
(5x_3 - 5x_3, 5x_3, 5x_3 + 5x_3, x_3 + x_4) &= (0, 5x_3, 10x_3, x_3 + x_4).
\end{align*}

So $U = \{(0, 5t, 10t, t+s) : t, s \in F\}$.

\textbf{Zero vector:} $(0, 0, 0, 0) \in U$ (take $t = s = 0$).

\textbf{Closed under addition:} 
\begin{align*}
(0, 5t, 10t, t+s) + (0, 5t', 10t', t'+s') = (0, 5(t+t'), 10(t+t'), (t+t')+(s+s')) \in U.
\end{align*}

\textbf{Closed under scalar multiplication:}
\begin{align*}
\lambda(0, 5t, 10t, t+s) = (0, 5(\lambda t), 10(\lambda t), \lambda t + \lambda s) \in U.
\end{align*}

Therefore $U$ is a subspace of $F^4$.
\hfill $\square$

\section*{Problem 2}
\textbf{Problem.}
Verify all assertions about subspaces in Example 1.35.

\textbf{Solution.}
Example 1.35 consists of problems (a) through (e) below, which have been verified in the following sections.
\hfill $\square$

\section*{(a)}
\textbf{Claim.}
If $b \in F$, then
\[
U = \{(x_1,x_2,x_3,x_4) \in F^4 : x_3 = 5x_4 + b\}
\]
is a subspace of $F^4$ if and only if $b = 0$.

\textbf{Proof.}
The zero vector $(0,0,0,0)$ lies in $U$ if and only if
\[
0 = 5\cdot 0 + b \iff b = 0.
\]
If $b \neq 0$, then $U$ does not contain the zero vector and hence is not a subspace.

If $b = 0$, then
\[
U = \{x \in F^4 : x_3 = 5x_4\}
\]
is defined by a homogeneous linear equation. One checks directly that $U$ is closed under addition and scalar multiplication, hence is a subspace.
\hfill $\square$

\section*{(b)}
\textbf{Claim.}
The set of continuous real-valued functions on $[0,1]$ is a subspace of $\mathbb{R}^{[0,1]}$.

\textbf{Proof.}
Let
\[
U = \{f : [0,1] \to \mathbb{R} : f \text{ is continuous}\}.
\]
The zero function is continuous, so $0 \in U$. If $f,g \in U$, then $f+g$ is continuous, and if $a \in \mathbb{R}$ then $af$ is continuous. Thus $U$ is closed under addition and scalar multiplication, hence is a subspace.
\hfill $\square$

\section*{(c)}
\textbf{Claim.}
The set of differentiable real-valued functions on $\mathbb{R}$ is a subspace of $\mathbb{R}^{\mathbb{R}}$.

\textbf{Proof.}
Let
\[
U = \{f : \mathbb{R} \to \mathbb{R} : f \text{ is differentiable}\}.
\]
The zero function is differentiable, so $0 \in U$. If $f,g \in U$, then $f+g$ is differentiable with
\[
(f+g)' = f' + g'.
\]
If $a \in \mathbb{R}$, then $af$ is differentiable with
\[
(af)' = a f'.
\]
Thus $U$ is closed under addition and scalar multiplication, and hence is a subspace.
\hfill $\square$

\section*{(d)}
\textbf{Claim.}
The set of differentiable real-valued functions $f$ on $(0,3)$ such that $f'(2) = b$ is a subspace of $\mathbb{R}^{(0,3)}$ if and only if $b = 0$.

\textbf{Proof.}
Let
\[
U = \{f : (0,3) \to \mathbb{R} : f \text{ is differentiable and } f'(2) = b\}.
\]
If $U$ is a subspace, it must contain the zero function. Since $0'(2) = 0$, this forces $b = 0$.

Conversely, if $b = 0$, then
\[
U = \{f : f'(2) = 0\}.
\]
The zero function lies in $U$. If $f,g \in U$, then
\[
(f+g)'(2) = f'(2) + g'(2) = 0,
\]
so $f+g \in U$. If $a \in \mathbb{R}$, then
\[
(af)'(2) = a f'(2) = 0,
\]
so $af \in U$. Thus $U$ is a subspace.
\hfill $\square$

\section*{(e)}
\textbf{Claim.}
The set of all sequences of complex numbers with limit $0$ is a subspace of $\mathbb{C}^\infty$.

\textbf{Proof.}
Let
\[
U = \{(z_n)_{n\ge 1} \in \mathbb{C}^\infty : \lim_{n\to\infty} z_n = 0\}.
\]

Here $(z_n)_{n\ge 1}$ denotes an infinite sequence $(z_1, z_2, z_3, \ldots)$ of complex numbers, which we can think of as an infinite tuple. The space $\mathbb{C}^\infty$ is the set of all such sequences.

To show $U$ is a subspace, we verify three conditions:

\textbf{(i) Zero vector:} The zero sequence $(0, 0, 0, \ldots)$ satisfies
\[
\lim_{n\to\infty} 0 = 0,
\]
so the zero vector is in $U$.

\textbf{(ii) Closed under addition:} Suppose $(x_n)_{n\ge 1}, (y_n)_{n\ge 1} \in U$. This means $\lim_{n\to\infty} x_n = 0$ and $\lim_{n\to\infty} y_n = 0$. Their sum is the sequence
\[
(x_n + y_n)_{n\ge 1} = (x_1 + y_1, x_2 + y_2, x_3 + y_3, \ldots).
\]
By limit laws,
\[
\lim_{n\to\infty} (x_n + y_n) = \lim_{n\to\infty} x_n + \lim_{n\to\infty} y_n = 0 + 0 = 0,
\]
so $(x_n + y_n)_{n\ge 1} \in U$.

\textbf{(iii) Closed under scalar multiplication:} Let $a \in \mathbb{C}$ and $(x_n)_{n\ge 1} \in U$. The scalar multiple is
\[
(a x_n)_{n\ge 1} = (a x_1, a x_2, a x_3, \ldots).
\]
By limit laws,
\[
\lim_{n\to\infty} (a x_n) = a \cdot \lim_{n\to\infty} x_n = a \cdot 0 = 0,
\]
so $(a x_n)_{n\ge 1} \in U$.

Therefore $U$ is a subspace of $\mathbb{C}^\infty$.
\hfill $\square$

\section*{Problem 3}
\textbf{Claim.}
The set of differentiable functions $f$ on the interval $(-4,4)$ such that $f'(-1) = 3f(2)$ is a subspace of $\mathbb{R}^{(-4,4)}$.

\textbf{Proof.}
Let
\[
U = \{f : (-4,4) \to \mathbb{R} : f \text{ is differentiable and } f'(-1) = 3f(2)\}.
\]

\textbf{Zero vector:} The zero function satisfies $0'(-1) = 0 = 3 \cdot 0 = 3 \cdot 0(2)$, so $0 \in U$.

\textbf{Closed under addition:} If $f, g \in U$, then
\[
(f+g)'(-1) = f'(-1) + g'(-1) = 3f(2) + 3g(2) = 3(f+g)(2),
\]
so $f + g \in U$.

\textbf{Closed under scalar multiplication:} If $f \in U$ and $a \in \mathbb{R}$, then
\[
(af)'(-1) = af'(-1) = a \cdot 3f(2) = 3(af)(2),
\]
so $af \in U$.

Therefore $U$ is a subspace.
\hfill $\square$

\section*{Problem 4}
\textbf{Claim.}
Suppose $b \in \mathbb{R}$. The set of continuous real-valued functions $f$ on $[0,1]$ such that $\int_0^1 f = b$ is a subspace of $\mathbb{R}^{[0,1]}$ if and only if $b = 0$.

\textbf{Proof.}
Let
\[
U = \left\{f : [0,1] \to \mathbb{R} : f \text{ is continuous and } \int_0^1 f = b\right\}.
\]

If $U$ is a subspace, it must contain the zero function. Since $\int_0^1 0 = 0$, we must have $b = 0$.

Conversely, suppose $b = 0$. Then the zero function is in $U$. If $f, g \in U$, then
\[
\int_0^1 (f+g) = \int_0^1 f + \int_0^1 g = 0 + 0 = 0,
\]
so $f + g \in U$. If $a \in \mathbb{R}$ and $f \in U$, then
\[
\int_0^1 (af) = a\int_0^1 f = a \cdot 0 = 0,
\]
so $af \in U$. Therefore $U$ is a subspace when $b = 0$.
\hfill $\square$

\section*{Problem 5}
\textbf{Claim.}
$\mathbb{R}^2$ is not a subspace of the complex vector space $\mathbb{C}^2$.

\textbf{Proof.}
While $\mathbb{R}^2 \subseteq \mathbb{C}^2$ as sets, $\mathbb{R}^2$ is not closed under scalar multiplication by complex scalars. For example, $(1, 0) \in \mathbb{R}^2$ but
\[
i \cdot (1, 0) = (i, 0) \notin \mathbb{R}^2.
\]
Therefore $\mathbb{R}^2$ is not a subspace of $\mathbb{C}^2$ when viewed as a complex vector space.
\hfill $\square$

\section*{Problem 6}

\subsection*{(a)}
\textbf{Claim.}
$\{(a, b, c) \in \mathbb{R}^3 : a^3 = b^3\}$ is not a subspace of $\mathbb{R}^3$.

\textbf{Proof.}
Let $U = \{(a, b, c) \in \mathbb{R}^3 : a^3 = b^3\}$. Note that $(1, 1, 0) \in U$ and $(1, 1, 0) \in U$, but
\[
(1, 1, 0) + (1, 1, 0) = (2, 2, 0).
\]
We check: $2^3 = 8 = 2^3$, so actually $(2, 2, 0) \in U$.

Let's try another approach. Consider $(1, 1, 0) \in U$ (since $1^3 = 1^3$). Then
\[
2 \cdot (1, 1, 0) = (2, 2, 0).
\]
We have $2^3 = 8 = 2^3$, so this is also in $U$.

Actually, if $a^3 = b^3$ in $\mathbb{R}$, then $a = b$ (since $x \mapsto x^3$ is injective on $\mathbb{R}$). So
\[
U = \{(a, a, c) : a, c \in \mathbb{R}\}.
\]
This is clearly a subspace (it's a plane through the origin).

Wait, let me reconsider. The problem asks if this IS a subspace. Let me verify properly.

If $a^3 = b^3$ with $a, b \in \mathbb{R}$, then $(a-b)(a^2 + ab + b^2) = 0$. For real numbers, $a^2 + ab + b^2 = 0$ implies $a = b = 0$ (completing the square: $(a + b/2)^2 + 3b^2/4 \geq 0$ with equality iff $a = b = 0$). So either $a = b$ or $a = b = 0$.

Actually, that's not quite right. We have $a^3 = b^3 \iff a^3 - b^3 = 0 \iff (a-b)(a^2+ab+b^2) = 0$. Since $a^2 + ab + b^2 = (a+b/2)^2 + 3b^2/4 > 0$ unless $a = b = 0$, we must have $a = b$.

So $U = \{(a, a, c) : a, c \in \mathbb{R}\}$, which is indeed a subspace (a 2-dimensional plane).

Actually, I should be more careful. This is a subspace!

\textbf{Proof (corrected).}
For real numbers, $a^3 = b^3$ if and only if $a = b$ (since $x \mapsto x^3$ is strictly increasing). Thus
\[
U = \{(a, a, c) \in \mathbb{R}^3 : a, c \in \mathbb{R}\}.
\]
This is clearly a subspace: it contains $(0,0,0)$, and if $(a, a, c), (a', a', c') \in U$, then
\[
(a, a, c) + (a', a', c') = (a+a', a+a', c+c') \in U,
\]
and for $\lambda \in \mathbb{R}$,
\[
\lambda(a, a, c) = (\lambda a, \lambda a, \lambda c) \in U.
\]
Therefore $U$ is a subspace of $\mathbb{R}^3$.
\hfill $\square$

\subsection*{(b)}
\textbf{Claim.}
$\{(a, b, c) \in \mathbb{C}^3 : a^3 = b^3\}$ is not a subspace of $\mathbb{C}^3$.

\textbf{Proof.}
Over $\mathbb{C}$, the equation $a^3 = b^3$ does not imply $a = b$. For instance, let $\omega = e^{2\pi i/3}$ be a primitive cube root of unity. Then $(1, \omega, 0)$ satisfies $1^3 = 1 = \omega^3$, so $(1, \omega, 0) \in U$.

Consider scalar multiplication: $\omega(1, \omega, 0) = (\omega, \omega^2, 0)$. We need $\omega^3 = (\omega^2)^3 = \omega^6 = 1$, which is true. So this is in $U$.

Let's check closure under addition. We have $(1, 1, 0) \in U$ and $(1, \omega, 0) \in U$. Their sum is
\[
(1, 1, 0) + (1, \omega, 0) = (2, 1+\omega, 0).
\]
For this to be in $U$, we need $2^3 = (1+\omega)^3$, i.e., $8 = (1+\omega)^3$.

We have $\omega = e^{2\pi i/3} = -\frac{1}{2} + \frac{\sqrt{3}}{2}i$, so $1 + \omega = \frac{1}{2} + \frac{\sqrt{3}}{2}i = e^{i\pi/3}$.

Thus $(1+\omega)^3 = e^{i\pi} = -1 \neq 8$.

Therefore $(2, 1+\omega, 0) \notin U$, so $U$ is not closed under addition and hence not a subspace.
\hfill $\square$

\section*{Problem 7}
\textbf{Claim.}
If $U$ is a nonempty subset of $\mathbb{R}^2$ such that $U$ is closed under addition and under taking additive inverses (meaning $-u \in U$ whenever $u \in U$), then $U$ is a subspace of $\mathbb{R}^2$.

\textbf{Counterexample.}
This is false. Let $U = \mathbb{Z}^2 = \{(m, n) : m, n \in \mathbb{Z}\}$. Then $U$ is closed under addition and additive inverses, but $U$ is not closed under scalar multiplication: $(1, 0) \in U$ but $\frac{1}{2}(1, 0) = (\frac{1}{2}, 0) \notin U$.

Therefore $U$ is not a subspace of $\mathbb{R}^2$.
\hfill $\square$

\section*{Problem 8}
\textbf{Claim.}
There exists a nonempty subset $U$ of $\mathbb{R}^2$ that is closed under scalar multiplication but is not a subspace.

\textbf{Example.}
Let $U = \{(x, 0) : x \in \mathbb{R}\} \cup \{(0, y) : y \in \mathbb{R}\}$ be the union of the $x$-axis and $y$-axis.

$U$ is closed under scalar multiplication: if $(x, 0) \in U$ and $a \in \mathbb{R}$, then $a(x, 0) = (ax, 0) \in U$. Similarly for points on the $y$-axis.

However, $U$ is not closed under addition: $(1, 0) \in U$ and $(0, 1) \in U$, but $(1, 0) + (0, 1) = (1, 1) \notin U$.

Therefore $U$ is not a subspace.
\hfill $\square$

\section*{Problem 9}
\textbf{Claim.}
The set of periodic functions from $\mathbb{R}$ to $\mathbb{R}$ is a subspace of $\mathbb{R}^{\mathbb{R}}$.

\textbf{Proof.}
Let
\[
U = \{f : \mathbb{R} \to \mathbb{R} : \exists p > 0 \text{ such that } f(x+p) = f(x) \text{ for all } x\}.
\]

\textbf{Zero vector:} The zero function is periodic (with any period), so $0 \in U$.

\textbf{Closed under addition:} If $f, g \in U$ with periods $p_f$ and $p_g$ respectively, then $f + g$ is periodic with period $\text{lcm}(p_f, p_g)$ (or more generally, any common multiple). For any $x \in \mathbb{R}$ and $p = p_f \cdot p_g$,
\[
(f+g)(x+p) = f(x+p) + g(x+p) = f(x) + g(x) = (f+g)(x).
\]
So $f + g \in U$.

\textbf{Closed under scalar multiplication:} If $f \in U$ with period $p$ and $a \in \mathbb{R}$, then
\[
(af)(x+p) = af(x+p) = af(x) = (af)(x),
\]
so $af$ has period $p$ and thus $af \in U$.

Therefore $U$ is a subspace of $\mathbb{R}^{\mathbb{R}}$.
\hfill $\square$

\section*{Problem 10}
\textbf{Claim.}
If $V_1$ and $V_2$ are subspaces of $V$, then $V_1 \cap V_2$ is a subspace of $V$.

\textbf{Proof.}
Let $U = V_1 \cap V_2$.

\textbf{Zero vector:} Since $V_1$ and $V_2$ are subspaces, $0 \in V_1$ and $0 \in V_2$, so $0 \in V_1 \cap V_2 = U$.

\textbf{Closed under addition:} If $u, v \in U$, then $u, v \in V_1$ and $u, v \in V_2$. Since $V_1$ is a subspace, $u + v \in V_1$. Since $V_2$ is a subspace, $u + v \in V_2$. Therefore $u + v \in V_1 \cap V_2 = U$.

\textbf{Closed under scalar multiplication:} If $u \in U$ and $a \in F$, then $u \in V_1$ and $u \in V_2$. Since $V_1$ is a subspace, $au \in V_1$. Since $V_2$ is a subspace, $au \in V_2$. Therefore $au \in V_1 \cap V_2 = U$.

Thus $U = V_1 \cap V_2$ is a subspace of $V$.
\hfill $\square$

\section*{Problem 12}
\textbf{Claim.}
The union of two subspaces of $V$ is a subspace of $V$ if and only if one of the subspaces is contained in the other.

\textbf{Proof.}
Let $U$ and $W$ be subspaces of $V$.

($\Leftarrow$) If $U \subseteq W$, then $U \cup W = W$, which is a subspace. Similarly if $W \subseteq U$.

($\Rightarrow$) Suppose $U \cup W$ is a subspace. We'll show that $U \subseteq W$ or $W \subseteq U$.

Assume for contradiction that $U \not\subseteq W$ and $W \not\subseteq U$. Then there exist $u \in U \setminus W$ and $w \in W \setminus U$.

Since $U \cup W$ is a subspace and $u, w \in U \cup W$, we have $u + w \in U \cup W$. Thus either $u + w \in U$ or $u + w \in W$.

\textbf{Case 1:} $u + w \in U$. Since $u \in U$ and $U$ is a subspace, $-u \in U$. Then
\[
w = (u+w) + (-u) \in U,
\]
contradicting $w \notin U$.

\textbf{Case 2:} $u + w \in W$. Since $w \in W$ and $W$ is a subspace, $-w \in W$. Then
\[
u = (u+w) + (-w) \in W,
\]
contradicting $u \notin W$.

Both cases lead to contradictions, so our assumption was wrong. Therefore $U \subseteq W$ or $W \subseteq U$.
\hfill $\square$

\section*{Problem 13}
\textbf{Claim.}
The union of three subspaces of $V$ is a subspace of $V$ if and only if one of the subspaces contains the other two.

\textbf{Proof.}
Let $V_1, V_2, V_3$ be subspaces of $V$.

($\Leftarrow$) If (say) $V_1 \supseteq V_2$ and $V_1 \supseteq V_3$, then $V_1 \cup V_2 \cup V_3 = V_1$, which is a subspace.

($\Rightarrow$) Suppose $U = V_1 \cup V_2 \cup V_3$ is a subspace. We'll show one subspace contains the other two.

Assume for contradiction that no subspace contains the other two. Then by the previous problem's reasoning, we can find elements:
\begin{itemize}
\item $v_1 \in V_1 \setminus (V_2 \cup V_3)$
\item $v_2 \in V_2 \setminus (V_1 \cup V_3)$  
\item $v_3 \in V_3 \setminus (V_1 \cup V_2)$
\end{itemize}

Consider $v_1 + v_2 \in U$. Since $U$ is a subspace and $v_1, v_2 \in U$, we have $v_1 + v_2 \in V_i$ for some $i \in \{1, 2, 3\}$.

\textbf{Case 1:} $v_1 + v_2 \in V_1$. Then $v_2 = (v_1 + v_2) - v_1 \in V_1$, contradicting $v_2 \notin V_1$.

\textbf{Case 2:} $v_1 + v_2 \in V_2$. Then $v_1 = (v_1 + v_2) - v_2 \in V_2$, contradicting $v_1 \notin V_2$.

\textbf{Case 3:} $v_1 + v_2 \in V_3$. Now consider $v_1 + v_3 \in U$. By similar reasoning, $v_1 + v_3$ cannot be in $V_1$ or $V_3$, so $v_1 + v_3 \in V_2$. But then
\[
v_3 = (v_1 + v_3) - v_1 = (v_1 + v_3) - [(v_1 + v_2) - v_2] \in V_2,
\]
using that $V_2$ is closed under addition and additive inverses, contradicting $v_3 \notin V_2$.

All cases lead to contradictions, so one subspace must contain the other two.
\hfill $\square$

\section*{Problem 14}
\textbf{Problem.}
Suppose $U = \{(x, -x, 2x) \in F^3 : x \in F\}$ and $W = \{(x, x, 2x) \in F^3 : x \in F\}$. Describe $U + W$ using symbols, and also give a description of $U + W$ that uses no symbols.

\textbf{Solution.}
\textbf{Symbolic description:}
\[
U + W = \{(a, -a, 2a) + (b, b, 2b) : a, b \in F\} = \{(a+b, -a+b, 2a+2b) : a, b \in F\}.
\]
Letting $x = a + b$ and $y = b$, we get $a = x - y$ and the third coordinate is $2x$. The second coordinate is $-a + b = -(x-y) + y = 2y - x$. So
\[
U + W = \{(x, 2y-x, 2x) : x, y \in F\}.
\]

Alternatively, note that any element of $U + W$ has the form $(s, t, 2s)$ where $s = a + b$ and $t = -a + b = 2b - s$ can be any value. So:
\[
U + W = \{(x, y, 2x) : x, y \in F\}.
\]

\textbf{Description without symbols:} $U + W$ is the set of all vectors in $F^3$ whose third coordinate is twice the first coordinate. This is a plane through the origin in $F^3$.
\hfill $\square$

\section*{Problem 15}
\textbf{Claim.}
If $U$ is a subspace of $V$, then $U + U = U$.

\textbf{Proof.}
By definition, $U + U = \{u_1 + u_2 : u_1, u_2 \in U\}$.

($\subseteq$) If $v \in U + U$, then $v = u_1 + u_2$ for some $u_1, u_2 \in U$. Since $U$ is a subspace, $u_1 + u_2 \in U$. Thus $v \in U$.

($\supseteq$) If $u \in U$, then $u = u + 0$ where $u, 0 \in U$. Thus $u \in U + U$.

Therefore $U + U = U$.
\hfill $\square$

\section*{Problem 16}
\textbf{Claim.}
The operation of addition on subspaces of $V$ is commutative.

\textbf{Proof.}
Let $U$ and $W$ be subspaces of $V$. By definition,
\[
U + W = \{u + w : u \in U, w \in W\}.
\]
Since addition in $V$ is commutative, $u + w = w + u$ for all $u \in U$ and $w \in W$. Therefore
\[
U + W = \{w + u : w \in W, u \in U\} = W + U.
\]
\hfill $\square$

\section*{Problem 17}
\textbf{Claim.}
The operation of addition on subspaces of $V$ is associative.

\textbf{Proof.}
Let $V_1, V_2, V_3$ be subspaces of $V$. We need to show $(V_1 + V_2) + V_3 = V_1 + (V_2 + V_3)$.

($\subseteq$) If $v \in (V_1 + V_2) + V_3$, then $v = u + v_3$ where $u \in V_1 + V_2$ and $v_3 \in V_3$. Since $u \in V_1 + V_2$, we have $u = v_1 + v_2$ for some $v_1 \in V_1$ and $v_2 \in V_2$. Thus
\[
v = (v_1 + v_2) + v_3 = v_1 + (v_2 + v_3).
\]
Since $v_2 + v_3 \in V_2 + V_3$ (as $v_2 \in V_2$ and $v_3 \in V_3$), we have $v \in V_1 + (V_2 + V_3)$.

($\supseteq$) The argument is symmetric.

Therefore $(V_1 + V_2) + V_3 = V_1 + (V_2 + V_3)$.
\hfill $\square$

\section*{Problem 18}
\textbf{Claim.}
The zero subspace $\{0\}$ is the additive identity for subspace addition. A subspace $U$ has an additive inverse if and only if $U = \{0\}$.

\textbf{Proof.}
\textbf{Additive identity:} For any subspace $U$ of $V$,
\[
U + \{0\} = \{u + 0 : u \in U, 0 \in \{0\}\} = \{u : u \in U\} = U.
\]
Similarly $\{0\} + U = U$. Thus $\{0\}$ is the additive identity.

\textbf{Additive inverses:} Suppose $U$ has an additive inverse $W$, meaning $U + W = \{0\}$. 

Since $0 \in U$ and $0 \in W$, we have $0 = 0 + 0 \in U + W = \{0\}$, which is correct.

For any $u \in U$, we have $u = u + 0$ where $u \in U$ and $0 \in W$. Thus $u \in U + W = \{0\}$, so $u = 0$. Therefore $U = \{0\}$.

Conversely, if $U = \{0\}$, then $U + U = \{0\}$, so $U$ is its own additive inverse.
\hfill $\square$

\section*{Problem 19}
\textbf{Claim.}
If $V_1, V_2, U$ are subspaces of $V$ such that $V_1 + U = V_2 + U$, then it does not necessarily follow that $V_1 = V_2$.

\textbf{Counterexample.}
Let $V = \mathbb{R}^2$, $V_1 = \{(x, 0) : x \in \mathbb{R}\}$ (the $x$-axis), $V_2 = \{(x, x) : x \in \mathbb{R}\}$ (the line $y = x$), and $U = \mathbb{R}^2$.

Then $V_1 + U = \mathbb{R}^2 = V_2 + U$, but $V_1 \neq V_2$.
\hfill $\square$

\section*{Problem 20}
\textbf{Problem.}
Suppose $U = \{(x, x, y, y) \in F^4 : x, y \in F\}$. Find a subspace $W$ of $F^4$ such that $F^4 = U \oplus W$.

\textbf{Solution.}
Note that $U$ has basis $\{(1, 1, 0, 0), (0, 0, 1, 1)\}$, so $\dim U = 2$.

We need $W$ such that $U \cap W = \{0\}$ and $U + W = F^4$. This requires $\dim W = 2$ (since $\dim F^4 = 4$).

One choice is
\[
W = \{(a, b, 0, 0) : a, b \in F\} = \text{span}\{(1, 0, 0, 0), (0, 1, 0, 0)\}.
\]

Wait, let me check this. If $(x, x, y, y) \in U \cap W$, then the element also has the form $(a, b, 0, 0)$. This means $y = 0$ and $x = a = b$. So $(x, x, 0, 0)$ with any $x$ is in the intersection. This is not just $\{0\}$.

Let me try again. Let
\[
W = \{(a, 0, b, 0) : a, b \in F\} = \text{span}\{(1, 0, 0, 0), (0, 0, 1, 0)\}.
\]

If $(x, x, y, y) \in W$, then it has the form $(a, 0, b, 0)$. So $x = 0$ and $y = 0$. Thus $U \cap W = \{0\}$.

For any $(a, b, c, d) \in F^4$, we can write
\[
(a, b, c, d) = (b, b, d, d) + (a-b, 0, c-d, 0),
\]
where $(b, b, d, d) \in U$ and $(a-b, 0, c-d, 0) \in W$. Thus $F^4 = U + W$.

Therefore
\[
W = \{(a, 0, b, 0) : a, b \in F\}.
\]
\hfill $\square$

\section*{Problem 21}
\textbf{Problem.}
Suppose $U = \{(x, y, x+y, x-y, 2x) \in F^5 : x, y \in F\}$. Find a subspace $W$ of $F^5$ such that $F^5 = U \oplus W$.

\textbf{Solution.}
Note that $U = \text{span}\{(1, 0, 1, 1, 2), (0, 1, 1, -1, 0)\}$, so $\dim U = 2$ (these vectors are linearly independent).

We need $\dim W = 3$. One approach is to extend the basis of $U$ to a basis of $F^5$. We look for three more linearly independent vectors.

Consider $e_1 = (1, 0, 0, 0, 0)$. Check if it's in $U$: if $(1, 0, 0, 0, 0) = (x, y, x+y, x-y, 2x)$, then $x = 1$, $y = 0$, but then $x + y = 1 \neq 0$. So $e_1 \notin U$.

Consider $e_2 = (0, 1, 0, 0, 0)$. If $(0, 1, 0, 0, 0) = (x, y, x+y, x-y, 2x)$, then $x = 0$, $y = 1$, but $x + y = 1 \neq 0$. So $e_2 \notin U$.

Consider $e_3 = (0, 0, 1, 0, 0)$. If $(0, 0, 1, 0, 0) = (x, y, x+y, x-y, 2x)$, then $x = 0$, $y = 0$, but $x + y = 0 \neq 1$. So $e_3 \notin U$.

But we need to make sure $W \cap U = \{0\}$. Let me use a more systematic approach.

The elements of $U$ satisfy:
\begin{align*}
z_3 &= z_1 + z_2 \\
z_4 &= z_1 - z_2 \\
z_5 &= 2z_1
\end{align*}

So we can choose $W$ to be the set of vectors where $z_1 = z_2 = 0$. That is,
\[
W = \{(0, 0, a, b, c) : a, b, c \in F\}.
\]

Check: If $(x, y, x+y, x-y, 2x) \in W$, then $x = 0$ and $y = 0$, so $U \cap W = \{0\}$.

For any $(a, b, c, d, e) \in F^5$, we can write
\begin{align*}
(a, b, c, d, e) &= (a, b, a+b, a-b, 2a) + (0, 0, c-(a+b), d-(a-b), e-2a).
\end{align*}

The first part is in $U$ and the second part is in $W$. Therefore $F^5 = U \oplus W$ with
\[
W = \{(0, 0, a, b, c) : a, b, c \in F\}.
\]
\hfill $\square$

\section*{Problem 22}
\textbf{Problem.}
Suppose $U = \{(x, y, x+y, x-y, 2x) \in F^5 : x, y \in F\}$. Find three subspaces $W_1, W_2, W_3$ of $F^5$, none of which equals $\{0\}$, such that $F^5 = U \oplus W_1 \oplus W_2 \oplus W_3$.

\textbf{Solution.}
From the previous problem, we know $\dim U = 2$, so we need $\dim W_1 + \dim W_2 + \dim W_3 = 3$ with each $W_i \neq \{0\}$.

The simplest choice is three 1-dimensional subspaces. Using the previous problem's approach where elements of $U$ have their first two coordinates as free parameters:

\[
W_1 = \text{span}\{(0, 0, 1, 0, 0)\}
\]
\[
W_2 = \text{span}\{(0, 0, 0, 1, 0)\}
\]
\[
W_3 = \text{span}\{(0, 0, 0, 0, 1)\}
\]

To verify: Any element of $F^5$ can be uniquely written as
\[
(a, b, c, d, e) = (a, b, a+b, a-b, 2a) + (0, 0, c-(a+b), 0, 0) + (0, 0, 0, d-(a-b), 0) + (0, 0, 0, 0, e-2a),
\]
where the first term is in $U$, and the remaining terms are in $W_1, W_2, W_3$ respectively.

The intersection of any pair (or more) of these subspaces is $\{0\}$ since they involve different coordinate positions being nonzero.

Therefore
\[
W_1 = \text{span}\{(0, 0, 1, 0, 0)\}, \quad W_2 = \text{span}\{(0, 0, 0, 1, 0)\}, \quad W_3 = \text{span}\{(0, 0, 0, 0, 1)\}.
\]
\hfill $\square$

\end{document}
