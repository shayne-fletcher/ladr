\documentclass[11pt]{article}
\usepackage{amsmath,amssymb,amsthm}
\usepackage{geometry}
\geometry{margin=1in}

\title{Section 1C: Subspaces\\Worked Solutions}
\author{}
\date{}

\begin{document}
\maketitle

\section*{Problem Setup}
Let $F$ be a field and $V$ a vector space over $F$. For each part below, we determine whether the given set is a subspace and justify the claim.

\section*{Problem 1}
For each of the following subsets of $F^4$, determine whether it is a subspace of $F^4$.

\subsection*{(a)}
\textbf{Claim.}
\[
U=\{(x_1-x_2,\;x_2,\;x_1+x_2,\;x_3+x_4)\in F^4:\ x_1,x_2,x_3,x_4\in F\}
\]
is a subspace of $F^4$. In fact,
\[
U=\{(u_1,u_2,u_3,u_4)\in F^4:\ u_3=u_1+2u_2\}.
\]

\textbf{Proof.}
Define a linear map $T:F^4\to F^4$ by
\[
T(x_1,x_2,x_3,x_4)=(x_1-x_2,\;x_2,\;x_1+x_2,\;x_3+x_4).
\]
Then $U=\operatorname{im}(T)$, hence $U$ is a subspace.

To describe $U$ explicitly, let $T(x_1,x_2,x_3,x_4)=(u_1,u_2,u_3,u_4)$. From $u_2=x_2$ and $u_1=x_1-x_2$ we get $x_1=u_1+u_2$, so
\[
u_3=x_1+x_2=(u_1+u_2)+u_2=u_1+2u_2.
\]
Conversely, given any $(u_1,u_2,u_4)$, choosing $x_2=u_2$, $x_1=u_1+u_2$, $x_3=u_4$, $x_4=0$ yields
$T(x_1,x_2,x_3,x_4)=(u_1,u_2,u_1+2u_2,u_4)$.
So exactly those vectors with $u_3=u_1+2u_2$ occur.
\hfill $\square$

\subsection*{(b)}
\textbf{Claim.}
\[
U=\{(x_1-x_2,\;x_2,\;x_1+x_2,\;x_3+x_4)\in F^4:\ x_1,x_2,x_3,x_4\in F,\ x_3=0\}
\]
is a subspace of $F^4$, and in fact it is the same subspace as in part (a).

\textbf{Proof.}
With $x_3=0$ we have the form $(x_1-x_2,\;x_2,\;x_1+x_2,\;x_4)$. As $x_4$ ranges over $F$, the fourth coordinate is arbitrary, and the first three coordinates satisfy the same relation $u_3=u_1+2u_2$ as in (a). Hence this set equals the $U$ from (a), so it is a subspace.
\hfill $\square$

\subsection*{(c)}
\textbf{Claim.}
\[
U=\{(x_1-x_2,\;x_2,\;x_1+x_2,\;x_3+x_4)\in F^4:\ x_1,x_2,x_3,x_4\in F,\ x_3=x_4\}
\]
is a subspace of $F^4$. Moreover,
\[
U=\{(u_1,u_2,u_1+2u_2,\;2t): u_1,u_2,t\in F\}.
\]
In particular, if $\mathrm{char}(F)\neq 2$ then $2t$ ranges over all of $F$, so $U$ coincides with the subspace from (a); if $\mathrm{char}(F)=2$ then $2t=0$ and the fourth coordinate must be $0$.

\textbf{Proof.}
If $x_3=x_4=t$, then the fourth coordinate is $x_3+x_4=2t$, and the first three coordinates are as in (a), hence satisfy $u_3=u_1+2u_2$.
This is the image of the linear map $F^3\to F^4$ given by $(x_1,x_2,t)\mapsto (x_1-x_2,x_2,x_1+x_2,2t)$, so it is a subspace, with the stated description.
\hfill $\square$

\subsection*{(d)}
\textbf{Claim.}
\[
U=\{(x_1-x_2,\;x_2,\;x_1+x_2,\;x_3+x_4)\in F^4:\ x_1=x_2=5x_3\}
\]
is a subspace of $F^4$.

\textbf{Proof.}
Write $t=x_3$ and $s=x_4$. Then $x_1=x_2=5t$ and the vector becomes
\[
(5t-5t,\;5t,\;5t+5t,\;t+s)=(0,\;5t,\;10t,\;t+s).
\]
Thus
\[
U=\{(0,5t,10t,t+s): t,s\in F\},
\]
which is the image of the linear map $F^2\to F^4$, $(t,s)\mapsto (0,5t,10t,t+s)$. Hence $U$ is a subspace (for every field $F$; the dimension may depend on $\mathrm{char}(F)$).
\hfill $\square$

\section*{Problem 2 (Example 1.35)}
Verify all assertions about subspaces in Example 1.35.

\subsection*{(a)}
\textbf{Claim.}
If $b\in F$, then
\[
U=\{(x_1,x_2,x_3,x_4)\in F^4: x_3=5x_4+b\}
\]
is a subspace of $F^4$ iff $b=0$.

\textbf{Proof.}
$(0,0,0,0)\in U$ iff $0=5\cdot 0+b$, i.e.\ iff $b=0$. If $b\neq 0$, $U$ cannot be a subspace.

If $b=0$, then $U=\{x\in F^4: x_3=5x_4\}$ is defined by a homogeneous linear equation, hence is closed under addition and scalar multiplication, so it is a subspace.
\hfill $\square$

\subsection*{(b)}
\textbf{Claim.}
The set of continuous real-valued functions on $[0,1]$ is a subspace of $\mathbb{R}^{[0,1]}$.

\textbf{Proof.}
Let $U=\{f:[0,1]\to\mathbb{R}: f\ \text{is continuous}\}$. The zero function is continuous, so $0\in U$. If $f,g\in U$, then $f+g$ is continuous; if $a\in\mathbb{R}$, then $af$ is continuous. Thus $U$ is closed under addition and scalar multiplication, hence is a subspace.
\hfill $\square$

\subsection*{(c)}
\textbf{Claim.}
The set of differentiable real-valued functions on $\mathbb{R}$ is a subspace of $\mathbb{R}^{\mathbb{R}}$.

\textbf{Proof.}
Let $U=\{f:\mathbb{R}\to\mathbb{R}: f\ \text{is differentiable}\}$. The zero function is differentiable, so $0\in U$. If $f,g\in U$, then $f+g$ is differentiable and $(f+g)'=f'+g'$. If $a\in\mathbb{R}$, then $af$ is differentiable and $(af)'=af'$. Hence $U$ is closed under addition and scalar multiplication and is a subspace.
\hfill $\square$

\subsection*{(d)}
\textbf{Claim.}
The set of differentiable real-valued functions $f$ on $(0,3)$ such that $f'(2)=b$ is a subspace of $\mathbb{R}^{(0,3)}$ iff $b=0$.

\textbf{Proof.}
Let $U=\{f:(0,3)\to\mathbb{R}: f\ \text{differentiable and}\ f'(2)=b\}$. If $U$ is a subspace then it contains the zero function, so $0'(2)=0=b$, hence $b=0$.

Conversely, if $b=0$, then $U=\{f: f'(2)=0\}$. If $f,g\in U$, then $(f+g)'(2)=f'(2)+g'(2)=0$, so $f+g\in U$. If $a\in\mathbb{R}$, then $(af)'(2)=af'(2)=0$, so $af\in U$. Thus $U$ is a subspace.
\hfill $\square$

\subsection*{(e)}
\textbf{Claim.}
The set of all sequences of complex numbers with limit $0$ is a subspace of $\mathbb{C}^{\infty}$.

\textbf{Proof.}
Let
\[
U=\{(z_n)_{n\ge 1}\in \mathbb{C}^{\infty}:\ \lim_{n\to\infty} z_n=0\}.
\]
The zero sequence has limit $0$, so $0\in U$. If $(x_n),(y_n)\in U$, then by limit laws,
\[
\lim_{n\to\infty}(x_n+y_n)=\lim x_n+\lim y_n=0+0=0,
\]
so $(x_n+y_n)\in U$. If $a\in\mathbb{C}$ and $(x_n)\in U$, then
\[
\lim_{n\to\infty}(a x_n)=a\lim x_n=a\cdot 0=0,
\]
so $(a x_n)\in U$. Hence $U$ is a subspace.
\hfill $\square$

\section*{Problem 3}
\textbf{Claim.}
The set of differentiable functions $f$ on $(-4,4)$ such that $f'(-1)=3f(2)$ is a subspace of $\mathbb{R}^{(-4,4)}$.

\textbf{Proof.}
Let $U=\{f:(-4,4)\to\mathbb{R}: f\ \text{differentiable and}\ f'(-1)=3f(2)\}$.
The zero function satisfies the condition. If $f,g\in U$, then
$(f+g)'(-1)=f'(-1)+g'(-1)=3f(2)+3g(2)=3(f+g)(2)$, so $f+g\in U$.
If $a\in\mathbb{R}$, then $(af)'(-1)=af'(-1)=3a f(2)=3(af)(2)$, so $af\in U$.
Thus $U$ is a subspace.
\hfill $\square$

\section*{Problem 4}
\textbf{Claim.}
Fix $b\in\mathbb{R}$. The set of continuous $f$ on $[0,1]$ such that $\int_0^1 f=b$ is a subspace of $\mathbb{R}^{[0,1]}$ iff $b=0$.

\textbf{Proof.}
If $U=\{f\ \text{continuous}:\int_0^1 f=b\}$ is a subspace, it contains the zero function, hence
$0=\int_0^1 0=b$, so $b=0$.
Conversely, if $b=0$, then for $f,g\in U$,
$\int_0^1(f+g)=\int_0^1 f+\int_0^1 g=0$, and for $a\in\mathbb{R}$,
$\int_0^1(af)=a\int_0^1 f=0$, so $U$ is closed under addition and scalar multiplication and hence a subspace.
\hfill $\square$

\section*{Problem 5}
\textbf{Claim.}
$\mathbb{R}^2$ is not a subspace of the complex vector space $\mathbb{C}^2$.

\textbf{Proof.}
As a subset, $\mathbb{R}^2\subseteq\mathbb{C}^2$, but it is not closed under scalar multiplication by complex scalars.
For example, $(1,0)\in\mathbb{R}^2$ but $i(1,0)=(i,0)\notin\mathbb{R}^2$.
\hfill $\square$

\section*{Problem 6}
\subsection*{(a)}
\textbf{Claim.}
\[
U=\{(a,b,c)\in\mathbb{R}^3: a^3=b^3\}
\]
\emph{is} a subspace of $\mathbb{R}^3$. In fact $U=\{(t,t,s): t,s\in\mathbb{R}\}$.

\textbf{Proof.}
Over $\mathbb{R}$, the map $x\mapsto x^3$ is injective, so $a^3=b^3$ implies $a=b$. Hence
$U=\{(t,t,s): t,s\in\mathbb{R}\}$.
This set contains $(0,0,0)$, is closed under addition and scalar multiplication, and is therefore a subspace.
\hfill $\square$

\subsection*{(b)}
\textbf{Claim.}
\[
U=\{(a,b,c)\in\mathbb{C}^3: a^3=b^3\}
\]
is \emph{not} a subspace of $\mathbb{C}^3$.

\textbf{Proof.}
Let $\omega=e^{2\pi i/3}$, so $\omega^3=1$ and $\omega\neq 1$.
Then $(1,1,0)\in U$ and $(1,\omega,0)\in U$ (since $1^3=\omega^3=1$).
If $U$ were a subspace it would be closed under addition, so their sum
$(2,1+\omega,0)$ would lie in $U$.
But $1+\omega=-\omega^2$ (because $1+\omega+\omega^2=0$), hence
\[
(1+\omega)^3=(-\omega^2)^3=-\omega^6=-1\neq 8=2^3,
\]
so $(2,1+\omega,0)\notin U$. Therefore $U$ is not closed under addition and is not a subspace.
\hfill $\square$

\section*{Problem 7}
\textbf{Claim.}
If $U\subseteq\mathbb{R}^2$ is nonempty, closed under addition, and closed under additive inverses, then $U$ is a subspace of $\mathbb{R}^2$.

\textbf{Counterexample.}
Let $U=\mathbb{Z}^2=\{(m,n): m,n\in\mathbb{Z}\}$. It is nonempty, closed under addition and under additive inverses, but not closed under scalar multiplication:
$\tfrac12(1,0)=(\tfrac12,0)\notin U$.
So the claim is false.
\hfill $\square$

\section*{Problem 8}
\textbf{Claim.}
There exists a nonempty subset $U\subseteq\mathbb{R}^2$ closed under scalar multiplication that is not a subspace.

\textbf{Example.}
Let $U$ be the union of the two coordinate axes:
\[
U=\{(x,0):x\in\mathbb{R}\}\ \cup\ \{(0,y):y\in\mathbb{R}\}.
\]
Then $U$ is closed under scalar multiplication, but not under addition since
$(1,0),(0,1)\in U$ while $(1,1)\notin U$.
\hfill $\square$

\section*{Problem 9}
\textbf{Claim.}
The set of periodic functions $\mathbb{R}\to\mathbb{R}$ is a subspace of $\mathbb{R}^{\mathbb{R}}$.

\textbf{Proof.}
Let $U=\{f:\mathbb{R}\to\mathbb{R}: \exists p>0\ \forall x,\ f(x+p)=f(x)\}$.
The zero function is periodic. If $f,g\in U$ with periods $p_f,p_g$, then $p=p_fp_g$ is a common period, so $f+g$ is periodic. If $a\in\mathbb{R}$, then $af$ has the same period as $f$. Hence $U$ is a subspace.
\hfill $\square$

\section*{Problem 10}
\textbf{Claim.}
If $V_1,V_2$ are subspaces of $V$, then $V_1\cap V_2$ is a subspace of $V$.

\textbf{Proof.}
$0\in V_1\cap V_2$. If $u,v\in V_1\cap V_2$, then $u+v\in V_1$ and $u+v\in V_2$, hence $u+v\in V_1\cap V_2$. Similarly $au\in V_1\cap V_2$ for any scalar $a$. Thus $V_1\cap V_2$ is a subspace.
\hfill $\square$

\section*{Problem 12}
\textbf{Claim.}
The union of two subspaces is a subspace iff one is contained in the other.

\textbf{Proof.}
If $U\subseteq W$ then $U\cup W=W$ is a subspace (and similarly if $W\subseteq U$).

Conversely, suppose $U\cup W$ is a subspace and neither contains the other. Pick $u\in U\setminus W$ and $w\in W\setminus U$.
Then $u+w\in U\cup W$. If $u+w\in U$, then $w=(u+w)-u\in U$, contradiction. If $u+w\in W$, then $u=(u+w)-w\in W$, contradiction. Hence one must contain the other.
\hfill $\square$

\section*{Problem 13}
\textbf{Claim.}
The union of three subspaces of $V$ is a subspace iff one of them contains the other two.

\textbf{Proof.}
If (say) $V_1\supseteq V_2$ and $V_1\supseteq V_3$, then $V_1\cup V_2\cup V_3=V_1$ is a subspace.

Conversely, suppose $U=V_1\cup V_2\cup V_3$ is a subspace. If $V_1$ contains $V_2\cup V_3$ we are done, so assume not.
Then there exists $v_2\in V_2\setminus V_1$ or $v_3\in V_3\setminus V_1$; WLOG pick $v_2\in V_2\setminus V_1$.
Similarly, if $V_2$ contains $V_1\cup V_3$ we are done; otherwise pick $v_1\in V_1\setminus V_2$.
Then $v_1+v_2\in U$. It cannot lie in $V_1$ (else $v_2=(v_1+v_2)-v_1\in V_1$) and cannot lie in $V_2$ (else $v_1\in V_2$). Hence $v_1+v_2\in V_3$.
Now for any $x\in V_1$, we have $x+v_2\in U$. The same subtraction argument shows $x+v_2$ cannot lie in $V_1$ or $V_2$, so $x+v_2\in V_3$, hence $x=(x+v_2)-v_2\in V_3$. Thus $V_1\subseteq V_3$.
Similarly, for any $y\in V_2$, $v_1+y\in U$ forces $v_1+y\in V_3$, hence $y\in V_3$. So $V_2\subseteq V_3$.
Therefore $V_3$ contains $V_1$ and $V_2$.
\hfill $\square$

\section*{Problem 14}
Suppose $U=\{(x,-x,2x)\in F^3:x\in F\}$ and $W=\{(x,x,2x)\in F^3:x\in F\}$. Describe $U+W$.

\textbf{Solution.}
\[
U+W=\{(a,-a,2a)+(b,b,2b):a,b\in F\}=\{(a+b,\,-a+b,\,2a+2b):a,b\in F\}.
\]
Let $s=a+b$ and $t=-a+b$. Then $2s=2a+2b$ and every pair $(s,t)$ arises from some $a,b$.
Hence
\[
U+W=\{(s,t,2s): s,t\in F\}.
\]
In words: $U+W$ is the set of all vectors in $F^3$ whose third coordinate is twice the first.
\hfill $\square$

\section*{Problem 15}
\textbf{Claim.}
If $U$ is a subspace of $V$, then $U+U=U$.

\textbf{Proof.}
If $u_1,u_2\in U$, then $u_1+u_2\in U$, so $U+U\subseteq U$. Also $u=u+0$ shows $U\subseteq U+U$. Thus $U+U=U$.
\hfill $\square$

\section*{Problem 16}
\textbf{Claim.}
Subspace addition is commutative: $U+W=W+U$.

\textbf{Proof.}
$U+W=\{u+w\}$ and $u+w=w+u$, so $U+W=W+U$.
\hfill $\square$

\section*{Problem 17}
\textbf{Claim.}
Subspace addition is associative: $(V_1+V_2)+V_3=V_1+(V_2+V_3)$.

\textbf{Proof.}
Both sides equal $\{v_1+v_2+v_3:\ v_i\in V_i\}$ by associativity of vector addition.
\hfill $\square$

\section*{Problem 18}
\textbf{Claim.}
$\{0\}$ is the additive identity for subspace addition. A subspace has an additive inverse iff it is $\{0\}$.

\textbf{Proof.}
$U+\{0\}=U$ since $u+0=u$. If $U+W=\{0\}$, then for any $u\in U$ we have $u=u+0\in U+W=\{0\}$, so $U=\{0\}$. Conversely $\{0\}+\{0\}=\{0\}$.
\hfill $\square$

\section*{Problem 19}
\textbf{Claim.}
If $V_1,V_2,U$ are subspaces with $V_1+U=V_2+U$, it need not follow that $V_1=V_2$.

\textbf{Counterexample.}
Take $V=\mathbb{R}^2$, $U=\mathbb{R}^2$, $V_1=\{(x,0)\}$ and $V_2=\{(x,x)\}$. Then $V_1+U=U=V_2+U$ but $V_1\neq V_2$.
\hfill $\square$

\section*{Problem 20}
Suppose $U=\{(x,x,y,y)\in F^4:x,y\in F\}$. Find $W$ such that $F^4=U\oplus W$.

\textbf{Solution.}
Let
\[
W=\{(a,0,b,0)\in F^4: a,b\in F\}.
\]
If $(x,x,y,y)\in U\cap W$, then $(x,x,y,y)=(a,0,b,0)$ forces $x=0$ and $y=0$, so $U\cap W=\{0\}$.
Also every $(p,q,r,s)\in F^4$ decomposes as
\[
(p,q,r,s)=(q,q,s,s)+(p-q,0,r-s,0),
\]
with the first term in $U$ and the second in $W$. Hence $F^4=U\oplus W$.
\hfill $\square$

\section*{Problem 21}
Suppose $U=\{(x,y,x+y,x-y,2x)\in F^5:x,y\in F\}$. Find $W$ such that $F^5=U\oplus W$.

\textbf{Solution.}
Take
\[
W=\{(0,0,a,b,c)\in F^5: a,b,c\in F\}.
\]
If $(x,y,x+y,x-y,2x)\in W$, then $x=y=0$, hence $U\cap W=\{0\}$.
Also every $(p,q,r,s,t)\in F^5$ decomposes as
\[
(p,q,r,s,t)=(p,q,p+q,p-q,2p) + (0,0,r-(p+q),\,s-(p-q),\,t-2p),
\]
with the first term in $U$ and the second in $W$. Thus $F^5=U\oplus W$.
\hfill $\square$

\section*{Problem 22}
With $U$ as in Problem 21, find nonzero subspaces $W_1,W_2,W_3$ such that $F^5=U\oplus W_1\oplus W_2\oplus W_3$.

\textbf{Solution.}
Let
\[
W_1=\mathrm{span}\{(0,0,1,0,0)\},\quad
W_2=\mathrm{span}\{(0,0,0,1,0)\},\quad
W_3=\mathrm{span}\{(0,0,0,0,1)\}.
\]
Then $W=W_1\oplus W_2\oplus W_3=\{(0,0,a,b,c)\}$ as in Problem 21, so $F^5=U\oplus W_1\oplus W_2\oplus W_3$.
\hfill $\square$

\section*{Problem 23}
\textbf{Claim.}
If $V=V_1\oplus U$ and $V=V_2\oplus U$, it need not follow that $V_1=V_2$.

\textbf{Counterexample.}
Let $V=F^2$, $U=\{(0,y):y\in F\}$, $V_1=\{(x,0):x\in F\}$, and $V_2=\{(x,x):x\in F\}$.
Then $F^2=V_1\oplus U$ since $(a,b)=(a,0)+(0,b)$ and $V_1\cap U=\{0\}$.
Also $F^2=V_2\oplus U$ since $(a,b)=(a,a)+(0,b-a)$ and $V_2\cap U=\{0\}$.
But $V_1\neq V_2$.
\hfill $\square$

\section*{Problem 24}
\textbf{Claim.}
Let $V_e$ be the even functions $\mathbb{R}\to\mathbb{R}$ and $V_o$ the odd functions. Then
$\mathbb{R}^{\mathbb{R}}=V_e\oplus V_o$.

\textbf{Proof.}
$V_e$ and $V_o$ are subspaces. If $f$ is both even and odd, then $f(x)= -f(x)$ for all $x$, so $f=0$, hence $V_e\cap V_o=\{0\}$.
For any $f:\mathbb{R}\to\mathbb{R}$ define
\[
f_e(x)=\frac{f(x)+f(-x)}{2},\qquad f_o(x)=\frac{f(x)-f(-x)}{2}.
\]
Then $f_e$ is even, $f_o$ is odd, and $f=f_e+f_o$, giving $V_e+V_o=\mathbb{R}^{\mathbb{R}}$. Uniqueness follows from $V_e\cap V_o=\{0\}$.
\hfill $\square$

\end{document}
